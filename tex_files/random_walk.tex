\section{Queueing Processes as Regulated Random Walks}
\label{sec:queu-proc-as}


In this section, we provide an elegant construction of a queueing process $\{L_k\}$ based on a \emph{random walk} $\{Z_k\}$.
This serves two aims.
The first is to show that a characterization of the transient behavior of a queueing process is typically extremely hard.
Thus, in the rest of the book we will only study queueing systems in steady-state.
The second is to show that queueing theory is essentially based on concepts of fundamental interest in probability theory (the random walk), hence is strongly related to many other applications of random walks, such as in finance, inventory theory, and insurance theory.
Moreover, we can use tools of random walk to analyze queueing systems, such a characterization of the distribution of the time until an especially large queue length is reached; we refer the reader to literature for this.




\opt{solutionfiles}{
\subsection*{Theory and Exercises}
\Opensolutionfile{hint}
\Opensolutionfile{ans}
}



In the construction of queueing processes as set out in~\cref{sec:constr-discr-time} we are given two sequences of i.i.d.
random variables: the number of arrivals $\{a_k\}$ per period and the service capacities $\{c_k\}$, cf.,~\cref{eq:5}.
Observe that in~\cref{eq:5} the process $\{L_k\}$ shares a resemblance to a random walk $\{Z_k, k=0,1,\ldots\}$ with $Z_k$ given by
\begin{equation}\label{eq:44}
 Z_k = Z_{k-1} + a_k - c_k.
\end{equation}
To see that $\{Z_k\}$ is indeed a random walk, observe that $Z$ makes jumps of size $a_k-c_k, k=1,\ldots$, and $\{a_k-c_k\}$ is a sequence of i.i.d.
random variables since, by assumption, $\{a_k\}$ and $\{c_k\}$ are i.i.d.
Clearly, $\{Z_k\}$ is `free', i.e., it can take positive and negative values, but $\{L_k\}$ is restricted to the non-negative integers.

\begin{exercise}\clabel{ex:l-133}
Show that $L_k$ satisfies the relation
\begin{equation}\label{eq:reich1}
 L_k = Z_k - \min_{1\leq i \leq k} Z_i\wedge 0,
\end{equation}
where $Z_k$ is defined by the above random walk and
we write $a\wedge b$ for $\min\{a,b\}$.
\begin{hint}
Note first that from the expression for $Z_k$,
 $a_k - c_k = Z_k - Z_{k-1}$. Use this to get
 $L_k = [L_{k-1} +Z_k- Z_{k-1}]^+$. Subtract $Z_k$ from both sides, use recursion and
 use subsequently,
\begin{align*}
&\max\{\max\{a,b\}, c\} = \max\{a,b,c\}, \\
&L_0 = Z_0, \\
&\max\{-a, -b \} = -\min\{a,b\}.
\end{align*}
\end{hint}
\begin{solution}
Note first that from the expression
for $Z_k$, $a_k - c_k = Z_k - Z_{k-1}$. Using this in the recursion
for $L_k$, we get
\begin{equation*}
 L_k = [L_{k-1} +Z_k- Z_{k-1}]^+,
\end{equation*}
thus, 
\begin{equation*}
 L_k - Z_{k} = \max\{L_{k-1} - Z_{k-1}, -Z_k\}.
\end{equation*}
From this, using recursion and the hints, we see that
\begin{equation*}
 \begin{split}
 L_k - Z_{k} 
% &= \max\{L_{k-1} - Z_{k-1}, -Z_k\} \\
&= \max\{\max\{L_{k-2} - Z_{k-2}, -Z_{k-1}\}, -Z_k\} \\
&= \max\{L_{k-2} - Z_{k-2}, -Z_{k-1}, -Z_k\} \\
&= \max\{L_{0} - Z_{0}, -Z_1, \ldots, -Z_k\} \\
&= \max\{0, -Z_1, \ldots, -Z_k\} \\
&= - \min\{0, Z_1, \ldots, Z_k\}.
 \end{split}
 \end{equation*}
 For further discussion, if you are interested, see~\citet{baccelli88:_sampl_m_m}.
\end{solution}
\end{exercise}

This recursion for $L_k$ leads to really interesting graphs.
In~\cref{fig:random_bernoulli} we take $a_k \sim B(0.3)$, i.e., $a_k$ is Bernoulli distributed with success parameter $p=0.3$, i.e., $\P{a_k = 1} = 0.3 = 1- \P{a_k=0}$, and $c_k \sim B(0.4)$.
In~\cref{fig:random_walk}, $a_k\sim B(0.49)$ and the random walk is constructed as
\begin{equation}\label{eq:51}
 Z_k = Z_{k-1} + 2 a_k -1.
\end{equation}
Thus, if $a_k=1$, the random walk increases by one step, while if $a_k=0$, the random walk decreases by one step, so that $Z_k \neq Z_{k-1}$ always. Observe that this is slightly different from a random walk that satisfies~\cref{eq:44}; there, $Z_{k}=Z_{k-1}$, if $a_k=c_k$.


\begin{figure}[ht]
 \centering
% see progs/reflected_random_walk.py
\input{progs/reflected_bernoulli_walk.tex}
%\input{reflected_random_walk.tex}
\caption{The upper panel shows a graph of the random walk $Z$. An
 upward pointing triangle corresponds to an arrival, a downward
 triangle to a potential service. The lower panel shows the queueing
 process $\{L_k\}$ as a random walk with reflection.}
\label{fig:random_bernoulli}
\end{figure}

\begin{figure}[ht]
 \centering
% see progs/reflected_random_walk.py
\input{progs/reflected_random_walk.tex}
\caption{Another example of a reflected random walk.}
\label{fig:random_walk}
\end{figure}


With~\cref{eq:reich1}, we see that a random walk $\{Z_k\}$ can be converted into a queueing process $\{L_k\}$, and we might try to understand the transient behavior of the latter by investigating the transient behavior of the former.
Suppose that $a_k\sim P(\lambda)$ and $c_k \sim P(\mu)$.

\begin{exercise}\clabel{ex:89}
  Show that if $\{a_k\}$ forms an i.i.d.
sequence of random variables all Poisson distributed $P(\lambda)$ then, $\sum_{j=1}^k a_j = P(\lambda k)$. 
\begin{hint}
Use~\cref{ex:1}.
\end{hint}
\begin{solution} Write $a$ for the common random variable. Then
  \begin{equation*}
    M_a(s) = \E{e^{sa}} = \sum_{j=0}^\infty e^{s j} e^{-\lambda j} \lambda^j/j! = e^{\lambda(e^{s}-1)}. 
  \end{equation*}
  Then with $X=\sum_{j=1}^k a_j$, and by independence, 
  \begin{equation*}
    M_X(s) = \left(M_a(s)\right)^k = e^{\lambda k (e^{s}-1)}. 
  \end{equation*}
  Hence, $X\sim P(\lambda k)$. 
\end{solution}
\end{exercise}


With the above exercise, 
\begin{equation*}
 Z_k = Z_0+N_{\lambda k} - N_{\mu k},
\end{equation*}
and we call $Z=\{Z_k\}$ the \emph{free} (discrete-time) $M/M/1$ queue as, contrary to the real $M/M/1$ queue, $Z$ can take negative values.

\begin{exercise}\clabel{ex:l-134}
 Show that when $n>m$ and $Z_0=m$, 
\begin{equation*}
 \P{Z_k=n}
= e^{-(\lambda+\mu)k} (\lambda k)^{n-m} \sum_{j=0}^\infty 
\frac{(\lambda\mu k^2)^j} {j!(n-m+j)!}.
\end{equation*}
\begin{solution}
\begin{equation}\label{eq:29}
 \begin{split}
 \P{Z_k=n}
&= \P{m+N_{\lambda k } - N_{\mu k} = n } \\ 
&= \P{N_{\lambda k}  - N_{\mu k}  = n - m } \\
&= \sum_{j=0}^\infty \P{N_{\lambda k}  =  n - m + j, N_{\mu k} =j}\\
&= \sum_{j=0}^\infty e^{-\lambda k} \frac{(\lambda k )^{n - m + j}}{(n-m+j)!} e^{-\mu k} \frac{(\mu k )^j}{j!} \\
&= e^{-(\lambda+\mu)k} (\lambda k)^{n-m} \sum_{j=0}^\infty  \frac{(\lambda\mu k^2)^j} {j!(n-m+j)!}.
 \end{split}
\end{equation}
\end{solution}
\end{exercise}


The solution of the above exercise shows that there is no simple function by which we can compute the transient distribution of this simple random walk $Z$.
Since a queueing process is typically a more complicated object (as we need to obtain $L$ from $Z$ via~\cref{eq:reich1}), our hopes of finding anything simple for the transient analysis of the $M/M/1$ queue should not be too high.
But the $M/M/1$ queue is about the simplest queueing system; other queueing systems are (much) more complicated.
We therefore give up the analysis of the transient behavior of queueing systems and henceforth contend ourselves with the analysis of queueing systems in the limit as $t\to\infty$.



\opt{solutionfiles}{
\Closesolutionfile{hint}
\Closesolutionfile{ans}
\subsection*{Hints}
\input{hint}
\subsection*{Solutions}
\input{ans}
}
%\clearpage

%%% Local Variables:
%%% mode: latex
%%% TeX-master: "../companion"
%%% End:
